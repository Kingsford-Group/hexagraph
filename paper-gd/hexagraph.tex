\documentclass{llncs}
%
\usepackage{makeidx}
\usepackage{algorithm}
\usepackage[noend]{algpseudocode}
\usepackage{graphicx}
\usepackage{booktabs}
\usepackage{xspace}
\usepackage{amsmath}
%
\begin{document}
%
\title{Drawing of sparse graphs via hexagonal tilings and iterative linear assignment}
%
\titlerunning{Drawing via hexagonal tilings}
%
\author{Carl Kingsford\inst{1}}
%
\authorrunning{Carl Kingsford}
%
\institute{Carnegie Mellon University, Pittsburgh PA 15213, USA,\\
\email{carlk@cs.cmu.edu},\\ 
WWW home page: \texttt{http://www.cs.cmu.edu/\homedir ckingsf}
}

\maketitle

\begin{abstract}
%
We present a method for drawing graphs that assigns nodes to tiles in a
hexagonal planar tiling. Adjacencies of the tiles are used to represent node
adjacencies in order to reduce the number of  edges that must be drawn explictly.
We show how an alternative method of drawing the tiles can be used to relax the
requirement that nodes assigned to touching tiles be adjacent in the input
graph.  We present an approach to finding a layout that attempts to minimize the number
of explicit edges drawn and to maximize other aesthetic qualities. The approach is based 
on the combination of several heuristics, the most central of which is the
repeated solution of a maximum weighted bipartite matching to find a good
assignment of nodes to tiles. The technique is shown to produce compact,
readable drawings for sparse graphs.
%
\end{abstract}

%-----------------------------------------------------------------------
\section{Introduction}

Drawing edges so that they can be easily tracked, so that adjacencies in a
graph can be quickly seen, and so that edges do not occlude nodes or each other
are central challenges in graph drawing. Various solutions to this problem have
been developed (e.g.\@, minimizing edge crossings~\cite{crossing} and using edge
bundling~\cite{bundling}, to name just two of many). Here, we consider an
approach that represents some edges as adjacencies between tiles in a
hexagonal, planar tiling. Because it is not always possible to embed a graph
into the tiling, the resulting drawings will contain a mixture of edges
represented by tile adjacencies and those represented by curved line segments.

Formally, we investigate the following problem:
%
\begin{problem}\label{prob:1}
%
Given a graph $G = (V,E)$ and a hexagonal tiling $\mathcal{T} = (C, F)$, where
$C$ are tiles and $F \subseteq C\times C$ are adjacencies between tiles, find a
function $f : V \mapsto C$ such that as many edges in $E$ as possible are mapped to
adjacent tiles in $F$. In other words, find $f$ to maximize
$\left|\left\{\{f(u),f(v)\} \in F : \{u,v\} \in E\right\}\right|$.
%
\end{problem}
%
Another way to view Problem~\ref{prob:1} is as a maximum graph homomorphism
problem restricted to the case of a hexagonal tiling. Consider $\mathcal{T}$ to be the
graph representing adjacencies of tiles. We seek a mapping $f$ from the nodes
of $G$ to the nodes of $\mathcal{T}$ so that for as many edges $\{u,v\} \in E$
as possible, we have $\{f(u),f(v)\}\in F$.

\begin{figure}\centering
%
\includegraphics[width=0.3\textwidth]{hexagon-geometry}
%
\caption{Hexagon tiling showing that the adjacency between any tile $A$ and any
number of its neighbors (here, just neighbor $6$) can be removed
by pulling an edge of the hexagon inward. This change does not affect 
any of the other tile adjacencies.}\label{fig:inward}
%
\end{figure}

Notice we do not require that $\{f(u),f(v)\}\in \mathcal{T} \implies \{u,v\}
\in G$. This is because hexagonal tilings have the property that any side of
the hexagonal tile can be moved inward independently without affecting any
other adjacencies of the tile (Figure~\ref{fig:inward}). This allows any
non-adjacent nodes to be assigned to adjacent tiles without introducing false
edges, and permits the creation of very compact drawings.  While
Problem~\ref{prob:1} could be defined for any planar tiling, this property of
hexagonal tiling (and other tilings) makes it particularly attractive. They are also
attractive because they are symmetric, fill the plane, and each tile has a
relatively large number of adjacencies (they are the largest regular polygon that tiles the plane). In addition, hexagons, if not constrained to a grid or uniform size are sufficient to represent any planar graph via adjacencies~\cite{thomassen}. Further, common motifs such as cliques,
bicliques, and stars show up directly in the layout, and nodes with high
clustering coefficient can also be seen as they will tend to have adjacent
tiles that are also adjacent. 

Figure~\ref{fig:florentine} provides an example of the type of drawing we will
produce.  These drawings reduce clutter by reducing the number of edges that
must be drawn and are intuitively readable. They are particularly
suited to situations when local adjacencies are of more interest than global structure. 

To solve Problem~\ref{prob:1} without
placing restrictions on the graph $G$ to be drawn, we develop an EM-like
algorithm that iteratively solves linear assignment instances (maximum
bipartite matching) to refine a mapping $f$ from vertices to tiles. We also present and test a
randomized approach that is used to avoid becoming trapped in poor local optima. We also test an approach to edge drawing and
coloring for edges that are not handled by tile adjacencies.

\paragraph{Related work.} The tessellation representation of planar
graphs~\cite{TT89b} is similar to the tiling approach described above in that
it omits directly drawing edges, though it is restricted to planar graphs,
requires nodes to be represented by rectangles, and --- arguably --- is not as
intuitive to read. Planar, 4-regular,  graphs can be drawn by associating
nodes with the intersections and touching points of circles the arcs of the
circles as edges~\cite{circles}. The visibility approach~\cite{TT86,vis96} to
drawing graphs provides another alternative in which edges can be implicitly
represented. Again, this approach has been restricted to planar graphs, and it can
produce drawings that are inorganic or unintuitive to read. Touching triangle graphs~\cite{touching} are the most similar to what we consider here: nodes are represented by triangles and edges by adjacencies of the triangles. There are are significant differences, however.  They have been applied to only some classes of graphs (outerplanar and some grid graphs); they do not support ``mixed'' drawings where some edges are not represented by triangle adjacencies; and the triangles do not form a tiling of the plane.
Prior work on ``hexagonal grid drawings''~\cite{hexagrid} refers to something different than our hexagonal tiling layouts: a triangular grid is used to constrain the layout of a node-and-edge drawing where nodes are placed on grid points and edges trace grid edges. This aids in reducing the number of edge bends and has applications in VLSI layout.  A similar iterative,
linear-assignment-based algorithm was given in~\cite{spin} for the different problem of finding
good orderings of the rows and columns of a matrix. 

\smallskip
This paper is not a theoretical paper in the sense that we give any bounds or
guarantees. Instead a number of algorithmic ideas are combined to produce a
practical heuristic for laying out graphs using the tiling scheme described
above. We show that in combination, these ideas can be applied to efficiently
produce hexagonal tilings for non-planar, sparse graphs that are high-quality
under several metrics such as small area, few node-edge crossings, and a
large number of edges implicitly represented.

\begin{figure}[t]\centering
%
\includegraphics[width=0.7\textwidth]{flor15}
%
\caption{The Florentine families graph ($n=15$, $m=20$)
from~\cite{Flor}.}\label{fig:florentine}
%
\end{figure}

%-----------------------------------------------------------------------
\section{Hexagonal Tiling Layout Approach}

\subsection{Iterative, randomized linear assignment for assigning nodes to
tiles}

\paragraph{Iterative linear assignment.}

To find an assignment that uses tile adjacency to represent as many edges as possible, we introduce an
EM-like, randomized procedure that iteratively improves an initial assignment.
This procedure, shown in Algorithm~\ref{alg:ilap}, computes the benefit of
moving each node $u$ to each tile assuming that all other nodes are fixed in
their current assignment. This results in a weighted, bipartite graph relating
nodes to tiles, and a maximum weight matching~\cite{hungarian} is found in this graph to choose
a new assignment based on these weights.  Algorithm~\ref{alg:ilap} gives an
overview of the procedure, and more detail on the various steps is given in the
sections below.

\begin{algorithm}[t]
\begin{algorithmic}[1]
\State Initialize $\sigma_0$.
\State $f'' \gets \Call{InitialAssignment}{G,\mathcal{T}}$
\Repeat
    \State $f \gets f''$
    \State $M_G(f,\sigma_t) \gets \Call{CreateMatchGraph}{G,\mathcal{T},f,\sigma_t}$\label{a:matchgraph}
    \State $f' \gets \Call{MaximumWeightMatching}{M_G(f,\sigma_t)}$\label{a:maxmatch}
    \State $f'' \gets \Call{PlaceUnplacedNodes}{G,\mathcal{T},f'}$\label{a:unplaced}
    \State $\sigma_{t+1} \gets \sigma_t + \Delta\sigma$
    \State $t \gets t+1$
\Until{$f'' = f$} 
\State $f \gets \Call{FloatIslands}{G,\mathcal{T},f}$\label{a:float}
\State $\Call{RouteUnsatisfiedEdges}{G,\mathcal{T},f}$\label{a:route}
\end{algorithmic}
%
\caption{Iterative, randomized linear assignment heuristic to solve
Problem~\ref{prob:1}.} \label{alg:ilap}
%
\end{algorithm}

\paragraph{Constructing a matching graph.}

Let $G = (V,E)$ be the graph to draw and let $f : V \rightarrow C$ be a 1-to-1
function from its nodes to the tiles $C$ of a planar tiling. We define a
weighted bipartite graph $M_G(f) = (V \cup C, A, w)$ with edge set $A$ and
their weights $w$ as follows.

\newcommand\neighT{\ensuremath\mathcal{N}_{\textit{\footnotesize tile}}}

Let $\neighT(c)$ be the set of tiles that are adjacent to $c$.  A tile $c$ is
\emph{blocked} for a node $u$ if there is some existing assignment to an
adjacent tile of another node that is not a neighbor of $u$. In other words, a
tile $c$ is blocked for $u$ if there is a $v\in V$ such that $f(v) \in
\neighT(c)$ but $\{u,v\}$ is not an edge of $G$.  Similarly, we say a tile is \emph{attractive} for $u$ if there is at
least one $v\in V$ such that $f(v) \in \neighT(c)$ and $\{u,v\} \in E$. Such
edges $\{u,v\}$ would be \emph{satisfied} by the assignment $f(u) = c$. Define
the satisfaction number $S_f(u,c)$ to be the number of satisfied edges incident
to $u$ if $u$ were assigned to tile $c$.

\newcommand\neighG{\ensuremath\mathcal{N}_G}

We add edge $(u,c)$ to $M_G$ if either $c$ is not blocked for $u$ or $c$ is
attractive for $u$ (line~\ref{a:matchgraph} in Algorithm~\ref{alg:ilap}). The
benefit of edge $(u,c)$ is set to
%
\begin{equation}\label{eqn:weight}
%
     w(u,c) = r\times S_f(u,c) + \sum_{v \in \neighG(u)} d(c,f(v)),
%
\end{equation}
%
where $r$ is a parameter specifying the reward for satisfying an edge,
$\neighG(u)$ are the neighbors of $u$ in $G$, and $d(c,d)$ is a function that
is decreasing as the planar distance between the tiles $c$ and $d$ increases.
This choice for $w(u,c)$ rewards both satisfying edges and  placing the
endpoints of unsatisfied edges nearby (because $d(c,d)$ will be high). Finally,
it also implicitly makes high-degree nodes more important because the summation
for high-degree nodes will contain more terms. Equation~\ref{eqn:weight} is but
one choice of possible weight function, and the algorithms below work for any
positive-valued function, though the quality of the layouts may differ. If the
edges $\{u,v\}$ of $G$ are weighted by their importance $i(u,v)$, the first term of~\eqref{eqn:weight}
can be changed to include benefit $ri(u,v)$ for each satisfied edge. For the experiments reported here, we set $r=1000$
and $d(c,d) = 500 / h_{cd}$ where $h_{cd}$ is a Euclidean distance between the
centers of tiles $c$ and $d$ divided by the hexagon radius (to make the units related to
number of tiles crossed).

\paragraph{Avoiding poor local minima.}

The iterative linear assignment procedure described above often gets stuck in
poor local minima or it oscillates between two poor solutions. To solve this,
we add an inertia effect that randomly keeps a subset of nodes in their current
tiles. Specifically, let $\sigma_t$ be a real number in $(0,1]$. On each iteration, for every $u$,
with probability $\sigma_t$, instead of adding the edges to $M_G$ described
above, we add the single edge $(u,f(u))$ with weight $r$. This strongly encourages a subset
of expected size $\sigma_t |V|$ of nodes to remain fixed; we call such nodes
\emph{inertia nodes}, and we denote the resulting bipartite graph
$M_G(f,\sigma_t)$. For the experiments here, we start with $\sigma_0 = 1/200$
and increase it by $\Delta\sigma = 1/400$ each iteration.

\paragraph{Handling unmatched nodes.}

Because $M_G(f,\sigma_t)$ is not complete and $f$ must be 1-to-1, the maximum
weight matching may not match every node to a tile. This is particularly likely
to happen with nodes that were chosen to be inertia nodes since they have only
a single tile to which they can match, and some other node may match to that
tile thus preventing the inertia node from being assigned anyplace.

To overcome this, we run a second matching just for the nodes that were
unmatched during the first round (line~\ref{a:unplaced}).  In this second
matching, there are edges $(u,c)$ between a node $u$ and any tile $c$ that is
not occupied or blocked for $u$ when fixing the assignments from the first
matching. In the experiments here, all the edge weights in this second matching are $1$, although more informative weights might be useful. Because the tiling field
is chosen to be large enough, the maximum matching in this second graph will
always include an assignment of every node to some tile.

% IDEA: only compute costs against inertia nodes.

%- (u,c) if either c is not blocked or c is attractive 
%    weight = 1000 * number of satisfied edges  + \sum_{neig} 500/celldist 
%- with decreasing inertia we add an edge $(u,f(u))$ of weight 1000.

\subsection{Local improvement by floating islands}

To improve the layout and more easily incorporate global desiderata, we end the
optimization with a phase of attempts at local improvement
(line~\ref{a:float}). We identify \emph{islands} of tiles that are connected
components of the graph after unsatisfied edges are removed. These islands are
continuous regions of the tiling that we will  move as a unit. To search for
local improvements, each island $I$ is considered in turn, and every possible
placement $I$ in the field of tiles is attempted while keeping the other
islands fixed in place. A placement is rejected if it would cause two islands to overlap. $I$ is left in the place that maximizes the quality $q$
of the resulting drawing, and the next island is considered. We repeatedly iterate through the islands until every island is in its locally optimal location for $q$ (because no island moves during a pass through the islands).

For the experiments here, $q$ is taken to be the area of the drawing so that
this local improvement step will tend to push islands together and create more
compact drawings. Although rotation of the islands could be executed as well,
we have not yet implemented it.

\subsection{Routing non-satisfied edges} 

\begin{figure}[t]\centering
%
\includegraphics[width=0.20\textwidth]{hexagon_graph}
%
\caption{Subgraph created for routing edges for each tile.}
\label{fig:routeg}
%
\end{figure}

We must draw a curve connecting $f(u)$ with $f(v)$ for any edge $\{u,v\}$ that
$f$ does not map to adjacent tiles. To do this (line~\ref{a:route}), we create
a graph $R$ by repeating the $K_6$ subgraph shown in Figure~\ref{fig:routeg}
for every tile.  The nodes of each $K_6$ are identified with the edges of the
hexagon tile, and adjacent tiles share nodes. Edges in $R$ that are within
tiles to which $f$ assigns some vertex of $G$  are given weight $10$,
while those in unassigned tiles are given weight $1$. In this way, edges drawn as curves
will avoid crossing occupied cells. Although these weights
are somewhat arbitrary, the $10$ is chosen sufficiently bigger than the
circumference of a tile so that routing around a tile is preferred over routing
through one.

For every unsatisfied edge $\{u,v\}\in E$, we find the shortest path in $R$ that
connects a vertex representing a side of tile $f(u)$ to a vertex representing a side of tile $f(v)$. The edge $\{u,v\}$
is then drawn by interpolating a curve along this shortest path. This shortest path
is called the \emph{route} of edge $\{u,v\}$. We also
experimented with a variant in which unsatisfied edges are routed in arbitrary
order and once an edge of $R$ is used for a route, its weight is increased by a
small amount; this tended to produce routes with complex and arbitrary curves,
selected simply to avoid reusing an edge of $R$. Hence, in the layouts below,
we allow arbitrary reuse of edges of $R$ without additional penalty.

\subsection{Coloring non-satisfied edges}

The above procedure often draws a number of routes using the same edge of $R$.
To visually distinguish such routes, we assign colors to routes so that if two
routes share an edge of $R$ they are assigned different colors. To do this, we
create a coloring graph $L = (V_L, E_L)$ where the vertices of  $L$ are the
routes (corresponding to unsatisfied edges of $G$), and two routes are
connected by an edge in $E_L$ if they use at least one of the same edges of
$R$. We then use a simple greedy approximation~\cite{greedycoloring} to find
the logical graph coloring of $L$ that uses the fewest colors. This gives a coloring of the
routes such that if two routes overlap, they are given different colors. Actual
colors are selected by choosing colors in hue-lightness-saturation space with
uniformly-spaced hues and near-constant (but slightly randomly perturbed)
lightness and saturation.

%-----------------------------------------------------------------------
\section{Results}

\subsection{Example tiled layouts}

The hexagonal tiling layout can produce very readable and attractive drawings,
as shown in Figure~\ref{fig:florentine}, which shows the relationship between
Florentine families from a classic social network analysis~\cite{Flor}. Here a
complex graph is drawn without edge crossings or node occlusion in a compact
area.

Figure~\ref{fig:tree} shows the layout of a 50-node, random tree with a
power-law degree distribution (computed via~\cite{networkx}). This graph is
planar, and nearly all edges are satisfied by tile adjacencies. The layout is
compact and readable. The long chain of degree-2 nodes at the root of the tree
(nodes 0 through 13) remains visible, and the high-degree nodes (26, 24, 25)
are easily recognizable. The ability to recognize this immediately as a tree,
however, is somewhat sacrificed.

\begin{figure}[t]
\centering
%
\includegraphics[width=0.9\textwidth]{tree50}
%
\caption{A random tree with 50 nodes.}\label{fig:tree}
%
\end{figure}

The number of times edges cross nodes is also typically much smaller for these
hexagonal tiling layouts. Compare, for example, Figure~\ref{fig:circuit} with
Figure~\ref{fig:circuit-neato}, which both show a circuit graph representing
the Apple II video controller from the 2010 Graph Drawing
Contest~\cite{gdcontest}. While the hexagonal tiling cannot avoid node-edge
crossings entirely, the layout produced by the force-directed approach of
graphviz (neato)~\cite{graphviz} includes far more such crossings. High-degree
nodes (such as \texttt{A5} and \texttt{A8}) are also more apparent in
Figure~\ref{fig:circuit} since they are drawn as flowers with a number of
petals.

Figures~\ref{fig:call} and~\ref{fig:angular} provide two more examples of
successful layouts produced by the hexagraph tiling approach.
Figure~\ref{fig:call} is a layout of the call graph for functions called by the
program that produced the tiling layout itself (only those functions that
represented more than 10\% of the execution of the program).
Figure~\ref{fig:angular} is the ``\texttt{angular}'' graph from the 2011 graph
drawing contest~\cite{gdcontest}. Note, for example, the excellent packing of
nodes $0-6$, $8-11$, and $13$ in Figure~\ref{fig:angular}.

\begin{figure}[p]\centering
%
\includegraphics[width=0.8\textwidth]{circuit7}
%
\caption{Circuit graph ($n=48$, $m=73$) from the 2010 Graph Drawing contest.}
\label{fig:circuit}
%
\end{figure}

\begin{figure}[p]\centering
%
\includegraphics[width=0.25\textwidth]{circuit7-neato}
%
\caption{Circuit graph of Figure~\ref{fig:circuit} drawn using
neato~\cite{graphviz}.}\label{fig:circuit-neato}
%
\end{figure}

\begin{figure}[p]\centering
%
\includegraphics[width=\textwidth]{call3}
%
\caption{Graph of function calls for the program which computes the hexagon
tiling layout ($n=52$, $m=54$).}\label{fig:call}
%
\end{figure}

\begin{figure}[p]\centering
%
\includegraphics[width=\textwidth]{angularA2}
%
\caption{The \texttt{angular A} graph ($n=48$, $m=102$) from the 2011 Graph
Drawing Contest~\cite{gdcontest}. The graph is planar.}\label{fig:angular}
%
\end{figure}

\subsection{Quality of layouts on several graphs}

% Sat Percent:
% 49.6 / 73 = 0.67945205479452
% 40.7 / 54 = 0.7537037037037
% 14 / 20 = 0.7
% 40.5 / 49 = 0.8265306122449
% 72 / 159 = 0.45283018867925
% 58.7 / 72 = 0.81527777777778
% 35.6 / 78 = 0.45641025641026
% 133.2 / 483 = 0.27577639751553

% Area Factor:
% 63 / 48 = 1.3125
% 72.3 / 52 = 1.39038461538462
% 17.8 / 15 = 1.18666666666667
% 65.7 / 50 = 1.314
% 83.7 / 62 = 1.35
% 74.2 / 57 = 1.30175438596491
% 42.3 / 34 = 1.24411764705882
% 201.1 / 141 = 1.42624113475177

% Node-Edge Crossings:
% 15.2 / 48 = 0.317
% 7.7 / 52 = 0.148
% 0.6 / 15 = 0.04
% 2.3 / 50 = 0.046
% 49.1 / 62 = 0.79193548387097
% 3.6 / 57 = 0.06315789473684
% 23.5 / 34 = 0.69117647058824
% 792.8 / 141 = 5.61702127659574

\newcommand\VarSatPercent{62\xspace}
\newcommand\VarAreaFactor{1.32\xspace}
\newcommand\VarMedianNodeEdgeCrossings{0.23\xspace}
\newcommand\VarLowIlapIter{26\xspace}
\newcommand\VarHighIlapIter{389\xspace}

Table~\ref{tbl:qual} shows the ability of the approach described above to
produce good layouts under several metrics. The graphs listed include those in
Figures~\ref{fig:florentine}, \ref{fig:tree}, \ref{fig:circuit},
\ref{fig:call}, and \ref{fig:angular}, as well as a dolphin association
network~\cite{dolphin}, the much-studied karate club social
network~\cite{karate}, and a graph representing the spatial structure of human
chromosome~10~\cite{lieberman}.   The most important metric, the one for which
we are most directly optimizing, is satisfying as many edges as possible via
tile adjacencies. In these tests, on average \VarSatPercent\% of the edges can be
represented by tile adjacencies, which represents a significant reduction in
the number of edges that must be drawn using curves. 

The number of edge colors needed is usually small, but it is larger than ideal
and can sometimes be quite large. This occurs when there are bottleneck
passages between assigned tiles or when there are many long-range routes to be
drawn. These high-traffic routing edges introduce some edge-edge occlusion that
--- even despite the different colors --- can make edges more difficult to
track. However, the use of alternative edge glyphs or edge offsets to reduce
confusion created when two edge routes use the same edge of $R$ could be
considered to mitigate this problem.

The greedy approximation for coloring works extremely well in the tested graphs. A
lower bound on the number of colors needed is the number of routes that pass
through the most-used edge of $R$. In only 3 instances, out of the 80 runs on
the graphs in Table~\ref{tbl:qual},  were the number of colors selected more than this
lower bound (in one instance each of \texttt{chrom10}, \texttt{Florentine}, and
\texttt{karate}, 138, 4, and 17 colors were used when the lower bounds were
136, 3, 14, respectively).  Even in these 3 instances, it may be that the greedy algorithm finds an
optimum coloring if the lower bound is not tight.

The minimum area required to display a graph of $n$ nodes is $n$ tiles. On average, the
layout is very compact, requiring an area of $\VarAreaFactor n$ tiles to display the nodes
(the actual area may be slightly larger to accommodate edge routes around the
circumference). This is evidence that the hexagon tiling approach allows very
compact drawings to be created, as expected.

Another advantage of this approach is that few node-edge crossings are
introduced. The median number of  node-edge crossings  present in the drawings
is \VarMedianNodeEdgeCrossings per node. These occur when either a node is
blocked on all sides by adjacent nodes, or when the endpoints of the edge are
close but a non-node-crossing route would need to nearly circumnavigate the
layout to reach an unblocked entry port of the node. Typically, the crossings
are not uniformly distributed across the nodes and some nodes are
``bottlenecks'' through which many edges pass.  This means that even when the
number of crossings per node is higher (e.g.\@ 0.79 for \texttt{dolphin}) the
graph usually remains very readable. 

\begin{table}[t]
%
\caption{Quality of computed layouts on several graphs, averaged over $10$
randomized runs. The number of nodes $n$ and number of edges $m$ are given.
\textbf{\# Satisfied} gives the edges that were satisfied by tile adjacencies.
\textbf{\# Colors} shows the number of colors required to color the routes.
\textbf{Area} gives the area of the drawing in terms of number of tiles.
\textbf{Node-Edge Crossings} are the number of times a non-satisfied edge
crosses the face of some node. \textbf{Route Lengths} gives the total length of
the routing of the unsatisfied edges (as the number of edges of $R$ that they
use).}\label{tbl:qual}
%
\centering
\begin{tabular}{lrrrrrrr}
\toprule
Graph                & $n$ & $m$ 
& \parbox{1.5cm}{\centering \# Edges\par Satisfied} 
& \parbox{1.5cm}{\centering\# Colors\par Needed} 
& Area 
& \parbox{1.8cm}{\centering Node-Edge\par Crossings} 
& \parbox{1.5cm}{\centering Route Lengths} \\
\midrule
\texttt{circuit}~\cite{gdcontest} 	& $48$ & $73$   & $49.6$ & $6.9$  & $63.0$   & $15.2$ & $182.0$ \\
\texttt{call} 		& $52$ & $54$   & $40.7$ & $5.2$  & $72.3$ & $7.7$  & $110.7$ \\
\texttt{Florentine}~\cite{Flor} & $15$ & $20$   & $14.0$ & $2.6$  & $17.8$ & $0.6$  & $29.4$ \\
\texttt{tree} 		& $50$ & $49$   & $40.5$ & $2.8$  & $65.7$ & $2.3$  & $55.5$ \\
\texttt{dolphin}~\cite{dolphin} 	& $62$ & $159$  & $72.0$ & $27.3$ & $83.7$ & $49.1$ & $776.8$\\ 
\texttt{flowchart}~\cite{gdcontest} 	& $57$ & $72$   & $58.7$ & $4.4$  & $74.2$ & $3.6$  & $113.7$\\
\texttt{karate}~\cite{karate} 	& $34$ & $78$   & $35.6$ & $12.4$ & $42.3$ & $23.5$ & $279.8$\\
\texttt{chrom10}~\cite{lieberman} 	& $141$ & $483$ & $133.2$ & $99.7$ & $201.1$ & $792.8$ & $3914.9$\\
\bottomrule
\end{tabular}
\end{table}

\subsection{Timing of the iterative linear assignment approach}

Table~\ref{tbl:time} gives running time information for the graphs of
Table~\ref{tbl:qual}. Most graphs of
$\approx 50$ nodes require around $5$ minutes of total computation time to
produce a good layout. This requires the solution of between \VarLowIlapIter
and \VarHighIlapIter maximum bipartite matching instances before the solution
converges to a stable assignment. In addition, a local optimal is found
typically after only a very few passes through the islands. Even a graph of
$483$ edges (\texttt{chrom10}) takes only $16$ minutes. These timings are
evidence that the approach described above can be practical. However, the
current code has not be optimized for speed: it is written in Python (using
igraph~\cite{igraph} to compute the maximum bipartite matching). Additional
engineering could likely significantly improve the speed at which layouts can
be produced.

\begin{table}[t]
%
\caption{Number of iterations of the linear assignment procedure (LAP) and the
greedy floating island improvements (Island), and the number of seconds for the
entire layout to be computed on a 3.4~GHz iMac in Python. Averaged
over $10$  runs.}\label{tbl:time}
%
\centering
\begin{tabular}{lrr@{$\;\;$}rrr}
\toprule
                     & \multicolumn{2}{c}{Size} & \multicolumn{2}{c}{Iterations} & \\
\cmidrule{2-3}\cmidrule(l){4-5}
Graph 				& \parbox{0.7cm}{\centering $n$}  & \parbox{0.7cm}{\centering $m$} & LAP  & Island  & Seconds \\
\midrule
\texttt{circuit} 	& $48$ & $73$   & $136.9$ & $2.4$ & $256.82$ \\
\texttt{call} 		& $52$ & $54$   & $122.3$ & $2.9$ & $252.61$ \\
\texttt{Florentine} & $15$ & $20$   & $46.7$  & $2.2$ & $19.63$ \\
\texttt{tree} 		& $50$ & $49$   & $64.4$  & $2.5$ & $149.99$ \\
\texttt{dolphin} 	& $62$ & $159$  & $314.1$ & $3.1$ & $571.25$ \\ 
\texttt{flowchart} 	& $57$ & $72$   & $84.3$  & $2.6$ & $218.98$ \\
\texttt{karate} 	& $34$ & $78$   & $233.9$ & $2.5$ & $221.50$ \\
\texttt{chrom10} 	& $141$ & $483$ & $374.8$ & $3.6$ & $955.84$ \\
\bottomrule
\end{tabular}
\end{table}

%-----------------------------------------------------------------------
\section{Discussion and Conclusion}

We have shown that the grid structure allows an efficient iterative, EM-like
procedure to find a compact layout with few node-edge crossings. However, there are many
ways in which the ``hexagraph'' layout presented here could be further
improved. The first is to allow non-satisfied edges to be routed through the
channels introduced when two non-adjacent nodes are placed in adjacent tiles.
This can be accomplished by creating a more complex routing graph $R$ that
includes edges for these channels. This will reduce the number of node-edge
crossings even further. 

Algorithmic improvements should also be explored to find even better
assignments of nodes to tiles. Currently, edge routing and node assignment are
considered almost independently: nodes are placed to satisfy edges and to
minimize the Euclidean distance between endpoints of unsatisfied nodes, but the
Euclidean distance can be a poor approximation to the distance an edge would
have to be routed around nodes. The length of routes could be incorporated into
the assignment costs in the maximum bipartite matching at the expense of
recomputing routes after each matching iteration. Improvements to the EM algorithm~\cite{em} could
also be considered.
 
We have shown that high-quality, readable, compact layouts that exploit node
adjacency to display the existence of edges can be computed efficiently using a
randomized, iterative linear assignment procedure, coupled with a local greedy
improvement phase. These algorithms produce drawings that are often  very
readable and are particularly suited for sparse graphs. We hope that this
general layout scheme will find uses in several application domains.

\section*{Acknowledgements}

This work has been partially funded by National Science Foundation
(CCF-1256087, CCF-1053918, and EF-0849899) and National Institutes of Health
(1R21AI085376 and 1R21HG006913). C.K. received support as an Alfred P. Sloan
Research Fellow.

\bibliographystyle{plain}
\bibliography{hexagraph}
\end{document}
