% This is LLNCS.DEM the demonstration file of
% the LaTeX macro package from Springer-Verlag
% for Lecture Notes in Computer Science,
% version 2.4 for LaTeX2e as of 16. April 2010
%
\documentclass{llncs}
%
\usepackage{makeidx}  % allows for indexgeneration
\usepackage[noend]{algpseudocode}
%
\begin{document}
%
\title{Drawing of near-planar graphs via hexagonal tilings}
%
\titlerunning{Drawing via hexagonal tilings}  % abbreviated title (for running head)
%                                     also used for the TOC unless
%                                     \toctitle is used
%
\author{Carl Kingsford\inst{1}}
%
\authorrunning{Carl Kingsford} % abbreviated author list (for running head)
%
\institute{Carnegie Mellon University, Pittsburgh PA 15214, USA,\\
\email{carlk@cs.cmu.edu},\\ WWW home page:
\texttt{http://www.cs.cmu.edu/\homedir ckingsf}
}

\maketitle              % typeset the title of the contribution

\begin{abstract}
The abstract should summarize the contents of the paper
using at least 70 and at most 150 words. It will be set in 9-point
font size and be inset 1.0 cm from the right and left margins.
There will be two blank lines before and after the Abstract. \dots
\keywords{computational geometry, graph theory, Hamilton cycles}
\end{abstract}

%-----------------------------------------------------------------------
\section{Introduction}


%-----------------------------------------------------------------------
\section{Layout Algorithms}

\subsection{Tile assignment as a quadratic program}

\subsection{Iterative, randomized linear assignment for assigning nodes to
tiles}

\paragraph{Iterative linear assignment.}

\begin{algorithmic}[1]
\State $\sigma_t \gets 1/200$
\State $f'' \gets \Call{InitialAssignment}{G,T}$
\Repeat
    \State $f \gets f''$
    \State $M_G(f,\sigma_t) \gets \Call{CreateMatchGraph}{f,\sigma_t}$
    \State $f' \gets \Call{MaximumWeightMatching}{M_G(f,\sigma_t)}$
    \State $f'' \gets \Call{PlaceUnplacedNodes}{f'}$
    \State $\sigma_t \gets \sigma_t + 1 / 200$
\Until{$f'' = f$} 
\end{algorithmic}


\paragraph{Choice of initial assignment.}

\paragraph{Constructing a matching graph.}

Let $G = (V,E)$ be the graph to draw and let $f : V \rightarrow T$ be a 1-to-1
function from its nodes to the tiles $T$ of a planar tiling. We define a
weighted bipartite graph $M_G(f) = (V \cup T, A, w)$ with edge set $A$ and
their weights $w$ as follows.

\newcommand\neighT{\ensuremath\mathcal{N}_{\textit{\footnotesize tile}}}
\newcommand\neighG{\ensuremath\mathcal{N}_{\textit{\footnotesize gr}}}

Let $\neighT(c)$ be the set of cells that are adjacent to $c$.  A cell $c$ is
\emph{blocked} for a node $u$ if there is some existing assignment of another
node that is not an neighbor of $u$ to an adjacent cell. In other words, a cell
$c$ is blocked for $u$ if there is a $v\in V$ such that $f(v) \in \neighT(c)$
but $\{u,v\}$ is not an edge of $G$. Define the blocking number $B_f(u,c)$ to
be the number of witnesses $v$ to $c$ being blocked for $u$ under $f$.
Similarly, we say a cell is \emph{attractive} for $u$ if there is at least one
$v\in V$ such that $f(v) \in \neighT(c)$ and $\{u,v\} \in E$. Such edges
$\{u,v\}$ would be \emph{satisfied} by the assignment $f(u) = c$. Define the
satisfaction number $S_f(u,c)$ to be the number of satisfied edges incident to
$u$.

We add edge $(u,c)$ to $M_G$ if either $c$ is not blocked for $u$ or $c$ is
attractive for $u$. The weight $w(u,c)$ is set to
%
\begin{equation}\label{eqn:weight}
%
     w(u,c) = R\times S_f(u,c) + \sum_{v \in \neighG(u)} d(c,f(v)),
%
\end{equation}
%
where $R$ is a parameter specifying the reward for satisfying an edge and
$d(c,d)$ is a function that is decreasing as the planar distance between the
cells $c$ and $d$ increases. For the experiments reported here, we take
$R=1000$ and $d(c,d) = 500 / h_{cd}$ where $h_{cd}$ is a Euclidean distance
between the centers of $c$ and $d$ divided by the hexagon radius. This choice
for $w(u,c)$ rewards satisfying edges and rewards placing the endpoints of
unsatisfied edges nearby (because $d(c,d)$ will be high). Finally, it also
implicitly makes high-degree nodes more important. Equation~\ref{eqn:weight} is
but one choice of possible weight function, and the algorithms below work for
any positive-valued function.

\paragraph{Avoiding poor local minima.}

The iterative linear assignment procedure described above often gets stuck in
poor local minima or it oscillates between two poor solutions. To solve this,
we add an inertia effect that randomly keeps a subset of nodes in their current
cells. Specifically, let $\sigma_t$ be a real number in $(0,1]$. For every $u$,
with probability $\sigma_t$, instead of adding the edges to $A$ described
above, we add the single edge $(u,f(u))$ with weight $1000$. This forces a
subset of expected size $\sigma_t |V|$ of nodes to remain fixed. For the
experiments here, we start with $\sigma_t = 1/200$ and increase it by $1/200$
each iteration.

\paragraph{Handling unmatched nodes.}

Because $A$ is not complete and $f$ must be 1-to-1, the maximum weight matching
may not match every node to a cell. This is particularly likely to happen with
nodes that were chosen to be inertia nodes since they have only a single cell
to which they can match, and some other node may match to that cell thus
preventing the inertia node from being assigned anyplace.

To overcome this, we run a second matching just for the nodes that were
unmatched during the first round.  In this second matching, we create a new
there are edges $(u,c)$ between a node $u$ and any cell $c$ that is not blocked
for $u$ when fixing the assignments from the first matching. All the edge
weights are $1$. Because the tiling field is chosen to be large enough, the
maximum matching in this second graph will always include an assignment of
every node to some cell.

% IDEA: only compute costs against inertia nodes.





%- (u,c) if either c is not blocked or c is attractive 
%    weight = 1000 * number of satisfied edges  + \sum_{neig} 500/celldist 
%- with decreasing inertia we add an edge $(u,f(u))$ of weight 1000.




\subsection{Relaxing the requirement that adjacent tiles encode edges}

\subsection{Routing non-satisfied edges}

\subsection{Coloring non-satisfied edges}

%-----------------------------------------------------------------------
\section{Results}

\subsection{Spectral approximation to the quadratic assignment problem}

\subsection{Timing of the iterative linear assignment approach}

\subsection{Quality of layout for near-planar graphs}

\subsection{Example tiled layouts}

%-----------------------------------------------------------------------
\section{Conclusion}

\end{document}
